
%% bare_conf.tex
%% V1.4b
%% 2015/08/26
%% by Michael Shell
%% See:
%% http://www.michaelshell.org/
%% for current contact information.
%%
%% This is a skeleton file demonstrating the use of IEEEtran.cls
%% (requires IEEEtran.cls version 1.8b or later) with an IEEE
%% conference paper.
%%
%% Support sites:
%% http://www.michaelshell.org/tex/ieeetran/
%% http://www.ctan.org/pkg/ieeetran
%% and
%% http://www.ieee.org/

%%*************************************************************************
%% Legal Notice:
%% This code is offered as-is without any warranty either expressed or
%% implied; without even the implied warranty of MERCHANTABILITY or
%% FITNESS FOR A PARTICULAR PURPOSE! 
%% User assumes all risk.
%% In no event shall the IEEE or any contributor to this code be liable for
%% any damages or losses, including, but not limited to, incidental,
%% consequential, or any other damages, resulting from the use or misuse
%% of any information contained here.
%%
%% All comments are the opinions of their respective authors and are not
%% necessarily endorsed by the IEEE.
%%
%% This work is distributed under the LaTeX Project Public License (LPPL)
%% ( http://www.latex-project.org/ ) version 1.3, and may be freely used,
%% distributed and modified. A copy of the LPPL, version 1.3, is included
%% in the base LaTeX documentation of all distributions of LaTeX released
%% 2003/12/01 or later.
%% Retain all contribution notices and credits.
%% ** Modified files should be clearly indicated as such, including  **
%% ** renaming them and changing author support contact information. **
%%*************************************************************************


% *** Authors should verify (and, if needed, correct) their LaTeX system  ***
% *** with the testflow diagnostic prior to trusting their LaTeX platform ***
% *** with production work. The IEEE's font choices and paper sizes can   ***
% *** trigger bugs that do not appear when using other class files.       ***                          ***
% The testflow support page is at:
% http://www.michaelshell.org/tex/testflow/



\documentclass[conference]{IEEEtran}

\usepackage{graphicx}
\usepackage{amsmath}
\usepackage{amssymb}
\usepackage{gensymb}
\usepackage{booktabs}

\graphicspath{ {C:/Users/Marc/Documents/Uni/van2016} }

\hyphenation{op-tical net-works semi-conduc-tor}

\renewcommand{\labelitemi}{$-$} 

\begin{document}

\title{Application of RCT to Describe Flight Passenger Flows}

\author{\IEEEauthorblockN{Marc Osswald}
\IEEEauthorblockA{Swiss International Air Lines Ltd.\\
Basel, CH\\
marc.osswald@swiss.com}
\and
\IEEEauthorblockN{Marcel Wehrle}
\IEEEauthorblockA{University of Fribourg\\
Fribourg, CH\\
marcel.wehrle@unifr.ch}
\and
\IEEEauthorblockN{Edy Portmann}
\IEEEauthorblockA{University of Bern\\
Bern, CH\\
edy.portmann@iwi.unibe.ch}}

\maketitle

\begin{abstract}
Fuzzy Cognitive Maps (FCMs) are a representation of concepts affecting one another. By connecting these concepts, FCMs are capable of showing dependencies and relationships between each other to a certain degree. Normally, these dependencies are numerically translated, which impedes interpretability for humans because they communicate using natural language. This paper seeks to turn the mathematical output of an FCM into natural language sentences by applying Restriction-Centered Theory (RCT) to enhance the possibilities of knowledge transfer for humans via an FCM. The proposed framework connects FCMs and the RCT to produce not only verbalized dependencies but also statements about the dependencies of FCMs. As a proof of concept, a use case is introduced, where Swiss Airline's connecting passenger flows are analyzed. The statements of the framework's output are verified by an expert at the company that owns the data.\\
Keywords: Fuzzy Cognitive Maps, Restriction-Centered Theory, Verbalization of dependencies
\end{abstract}

	
\IEEEpeerreviewmaketitle

\section{Introduction}
Dealing with language is simple and complicated at the same time. There is a set of rules stating how words have to be arranged in order to build grammatically correct sentences, i.e., by learning and following the grammar rules, one is able to speak the language correctly. The big challenge of using a language is to find the exact words that express what someone wants to communicate to their counterpart. However, it is not guaranteed that the listener will interpret these words in the same way and thus have the same semantic the speaker does. Rapaport \cite{Rapaport2003} states that communication is a negotiation of meanings. If two people do not have a common understanding of the concept behind a word, they will experience a misunderstanding: even if the syntax is the same, the semantic will be different. \\
In computer systems, the meaning of a sentence cannot be recognized unless taught by a human to do so. There are many ways to convert complex information into bits and bytes, but information is also supposed to flow in the other direction. Computer systems need to be increasingly able to extract the relevant information and throw unusable information away. Moreover, they need to set information into the right context to achieve a common understanding. By communicating their findings to a human, not only in digits but in complete sentences (i.e., natural language), the transfer of knowledge is simplified. The computer system incurs some part of the interpretation of the numerical expressions.\\
The sentences should be filled with meaningful words. Statements like \emph{"In the past three years, the maximum temperature on Easter Sunday was (5.1\degree C, 6.2\degree C, 2.3\degree C)"} are then replaced by \emph{"In the past three years, Easter Sunday was always a cold day"}. At the same time, the problem of subjectivity arises: if someone thinks that every temperature above 0\degree C does not deserve the linguistic label \emph{"cold"}, then he might be misled by the information. Even worse, if someone has their limit at 5\degree C, then even the word \emph{"always"} is not correct anymore.\\
This kind of difference in interpretation of words is an unsolved problem for computer systems. However, there are ways to avoid this strict allocation of content to a concept (i.e., fuzzy computation) that allow absolute values like temperatures to belong not just to a single concept but to different concepts.\\
After having introduced the concepts of FCMs and RCT in Chapter \ref{sec:theory}, this paper seeks to find a way of shaping FCMs out of large amounts of data and to phrase the contents into complete and meaningful sentences. To do so, a framework is presented (Chapter \ref{sec:framework}) and implemented in a use case (Chapter \ref{sec:usecase}). Finally, Chapter \ref{sec:lessons} presents the lessons learned from the use case, and Chapter \ref{sec:conclusion} provides a general conclusion.\\


\section{Theory of Fuzzy Cognitive Maps and Restriction Centered Theory}
\label{sec:theory}
This chapter presents the theoretical elements of the FCM (Section \ref{subsec:fcm}) and the RCT (Section \ref{subsec:rct}).
\subsection{Fuzzy Cognitive Maps}
\label{subsec:fcm}

FCMs are an extension of the concept of cognitive maps, where precise or abstract concepts can be interconnected \cite{kosko1986}. In cognitive maps, the concepts are represented by nodes and the interconnections by directed graphs. Additionally, the edges are used for representing the information on how a concept influences another. This information is reduced to a plus or minus, meaning that a concept exerts a positive or a negative influence on its counterpart.\\
Compared to cognitive maps, FCM are capable of statements about \emph{how much} concepts influence each other by introducing edge weights in the interval between -1 and 1. This leads to major flexibility and accuracy in representing complex systems.\\
According to Stach et al. \cite{stach2005}, an FCM can be developed either manually or computationally. To visualize the basic idea of an FCM, Fig. \ref{fig:cm} shows a fictional minimized FCM on how much the profit of a flight route from A to B is influenced by the number of passengers, the petrol price or the number of connecting passengers.

\begin{figure}[ht]
\includegraphics[width=3.5in]{img/ficFCM.png}
\caption{Influences of the profitability flight route A to B}
\label{fig:cm}
\end{figure}

FCMs are widely used to represent a collection of knowledge of experts expressed by causal weighted digraphs. The developed model can be also interpreted by non-field-specific experts. This facilitates the transfer of knowledge through a simple, visual mechanism. If sufficient field specific data are available, FCM can be developed directly on these data and verified by field-specific experts, allowing the FCM to model complex problems based on large amounts of data and reduce them to the essential causal dependencies. This becomes an increasingly useful property because the global available amounts of data are growing increasingly currently.\\
Developed FCMs can be used to \emph{explain} a complex system by showing the degrees of causal influences of relevant concepts for the system. They can \emph{predict} how changes of these causal influences can affect other concepts or the system as whole. They help decision makers to \emph{reflect} over a given situation and check where adjustments are needed, which can also influence the \emph{strategical planning}. Finally, they can \emph{visualize} a complex system by introducing a graphical interface \cite{papageorgiou2013}.
\vspace{2mm}
\newline
\emph{Formal Definition of FCMs}
\vspace{1mm}
\newline
A FCM \begin{math} F = [-1,1] \end{math} is a 4-tuple \begin{math} (N,E,C,f) \end{math}, where
\begin{enumerate}
\item \begin{math} N = {N_{1},N_{2},...,N_{n}} \end{math} is a set of $n$ concepts.
\item \begin{math} E: (N_{i},N_{j}) \rightarrow e_{ij} \end{math} is a function of \begin{math} N \times N \end{math} to \begin{math} K \end{math} associating \begin{math} e_{ij} \end{math} equal to zero if \begin{math} i = j \end{math}. Thus, \begin{math} E(N \times N) = (e_{ij}) \in K^{n \times n} \end{math} is a connection matrix.
\item \begin{math} C: N_{i} \rightarrow C_{i} \end{math} is a function whereby at each concept, \begin{math} N_{i} \end{math} associates the sequence of its activation degrees such as for \begin{math} t \in \mathbb{N}, C_{i}(t) \in L \end{math} given its activation degree at the moment \begin{math} t \end{math}. \begin{math} C(0) \in L^{n} \end{math} indicates the initial vector and specifies initial values of all concept nodes, and \begin{math} C(t) \in L^{n} \end{math} is a state vector at a certain iteration \begin{math} t \end{math}.
\item \begin{math} f: \mathbb{R} \rightarrow L \end{math} is a transformation function, which includes a recurring relationship on \begin{math} t \geqslant 0 \end{math} between \begin{math} C(t+1) \end{math} and \begin{math} C(t) \end{math}.
\end{enumerate}
where \begin{math} \mathbb{R} \end{math} is the set of real numbers and \begin{math} \mathbb{N} \end{math} denotes the set of natural numbers. \begin{math} K = [-1,1] \end{math} and \begin{math} L = [0,1] \end{math} \cite{stach2005}

The initiation of FCM starts with an initial vector as the input to a functional model (Equation \ref{eq:fm}, \cite{stach2005}); this results in a (initial) state vector. If over several iterations the sequence of state vectors is repeating, a steady state is reached.  
\begin{equation}\label{eq:fm}
\forall i \in \{1,...,n\},C_{i}(t+1) = f  \bigg(\sum_{i=1 \atop i \neq j}^{n} e_{ji}C_{j}(t)\bigg)
\end{equation}
The equation \ref{eq:fm} shows the multiplication of the initial vector with the corresponding weights. By applying a transformation function, these weights can be normalized in the range -1 to 1. To interpret an FCM in a first step, several indicators can be used (i.e., nodes can be classified into transmitter, receiver and ordinary, determined by the indegree and outdegree of each node \cite{ozesmi2004}).\\
The centrality of a node can simply be measured by summing up the outdegree and indegree. The next subsection introduces the concept of Restriction Centered Theory, which can be used to translate the output of an FCM into natural language. 

\subsection{Restriction Centered Theory}
\label{subsec:rct}
Fuzziness can be either expressed by using complex probabilistic models or, more simply, a single word. Again, an example of the chosen airline use case, \emph{"The flight lasts long"}, is a statement of a person that may be grasped differently even though there is a conjoint understanding about the concept of longevity. In brief, the RCT seeks to evaluate the duration of a flight that is perceived as being long.
\vspace{2mm}
\newline
\emph{General Idea behind  Restriction Centered Theory}
\vspace{1mm}
\newline
There are different concepts related to defining a range of answers to a specific question (e.g., \emph{"How true is X?"}). The range can be defined either as a numerical interval, a set with predefined attributes or a probability distribution. The fact that a word can be used in different contexts makes it complicated. Even worse, depending on the context, the same word can additionally have a different meaning. \\
The main concept of the RCT is the restriction. It contains a higher variety in its range than an interval, a set or a probability distribution and combines all these concepts: a restriction's range contains every value that can be imagined. The following example illustrates the variety of possible answers: if a person is asked a question such as \emph{"How profitable is the flight from A to B?"}, possible answers might be \emph{"It's 6.3 Billion US-Dollar per year"}, \emph{"It is more profitable than flight A to C"}, \emph{"It is not really profitable"} or \emph{"It is not as profitable as expected"}.\\
The first statement is palpable and can only be interpreted in one way. According to Zadeh \cite{zadeh2013}, the other statements are referred to as propositions. Their outstanding property is natural language. They contain fuzzy components, such as a predicate (i.e., \emph{fast, loud, high}), a quantifier (i.e., \emph{many, less}) and/or a probability (i.e., \emph{likely, eventually}). If a sentence includes a fuzzy predicate, the proposition is called \emph{zero-order}; if it contains any one or more of the components, the sentence is called \emph{first-order} fuzzy proposition.\\ 
Asking an artificial system how profitable the flight from A to B is, it can only interpret the first answer unless it is able to turn the proposition into a computable form. Applying RCT, this is achievable. \\
To depict a proposition in a mathematical form, the Meaning Postulate (MP) is used:
\begin{equation} \label{eq:mp}
p \rightarrow X isr R
\end{equation}
In this equation \ref{eq:mp}, 
$p$ is the proposition that contains a variable $X$ to be restricted, a restricting relation $R$, and the type of restricting relation $r$. The better \begin{math} X \end{math}, \begin{math} R \end{math} and \begin{math} r \end{math} are mathematically defined, the better a restriction can be computed. Computations with restrictions are pointed out in this example with fictitious numbers: \emph{"One out of about 5 flights routes is profitable. At Airline B, about 10 are profitable. What is the number of flight routes handled by Airline B?"} These two restrictions imply that Airline B has (more or less) flight routes. Humans are used to this kind of fuzzy reasoning. RCT offers an approach for solving these problems computationally.\\
A second important element of the RCT is the Canonical Form (CF). Basically, \begin{math} CF(p) \end{math} is the right-hand side of Equation \ref{eq:mp}, assigning the correct restriction type to the proposition.\\
The last concept is the Truth Postulate (TP). It computes the truth degree of a MP and is affiliated with the preciseness of the proposition. A truth degree can be phrased either numerically (i.e., a first-order truth value), or in natural language, which is known as a second-order truth value \cite{zadeh2013}.
\vspace{2mm}
\newline
\emph{Restrictions}
\vspace{1mm}
\newline
\label{subsubsec:restrictions}
Generally, natural language is used to express a restriction. The transformation of linguistic content into a calculable shape is called precisiation. An explanatory database (ED) with a collection of relations is used for the precisiation of the variables \begin{math} X \end{math} and \begin{math} R \end{math}, or the computation of the truth value of \begin{math} p \end{math}. The precisiated variables are called \begin{math} X^{*} \end{math}, \begin{math} R^{*} \end{math} and \begin{math}p^{*} \end{math}. Thus, \begin{math} X^{*}=f(ED) \end{math} and \begin{math} R^{*}=g(ED) \end{math}. The numerical truth value of \begin{math} p \end{math} and \begin{math}nt_{p} \end{math} can be depicted as \begin{math} nt_{p}=tr(ED) \end{math}, where \begin{math} tr \end{math} is called the truth function.\\
A restriction is subdivided into different types. The three main types are the possibilistic restriction, the probabilistic restriction and the Z-restriction, which is a combination of the two former types. 
If the type of the restricting relation $R$ is a singleton (e.g., $X = 5$), the restriction is characterized as singular, otherwise as nonsingular. Furthermore, the restriction may be direct, (i.e., the restricted variable is \begin{math} X \end{math}) or indirect (i.e., the restricted variable is a function of \begin{math} X \end{math}).\\
In the possibilistic restriction, \begin{math} R \end{math} is a fuzzy set, \begin{math} A \end{math}, to which \begin{math} X \end{math} belongs. The membership degree of \begin{math} X \end{math} to \begin{math} A \end{math} is calculated with the membership function, \begin{math} \mu_{A} \end{math}, of a possibility distribution.\\
The probabilistic restriction allows a statement about the sureness of a proposition. It is calculated with a density function of \begin{math} X \end{math}. Applied to natural language sentences, this means that the sureness of \emph{"usually"} is higher than the one of \emph{"sometimes"}. Probabilistic restrictions rarely occur exclusively in natural language. They are rather combined with possibilistic propositions and are thus Z-restrictions \cite{zadeh2013}. A Z-restriction contains natural language propositions, e.g., \emph{"Maybe Flight A to B is profitable"}, where \emph{"profitable"} is possibilistic and \emph{"maybe"} probabilistic.\\
The notation of the CF differs depending on the type of restriction. A possibilistic restriction is denoted as \begin{math} X is A \end{math}, where \begin{math} A \end{math} is the fuzzy set or relation. A probabilistic restriction is shown as \begin{math} X isp p \end{math}, where the first \begin{math} p \end{math} is the probability density function. The second \begin{math} p \end{math} stands for proposition in the MP. A Z-restriction is depicted as \begin{math} X iz Z \end{math}, where \begin{math} Z \end{math} is the combination of possibilistic and probabilistic restrictions and is defined as \begin{math} Z: Prob(X is A) is B \end{math}.\\
By combining both concepts, FCM and RCT, it is possible to formulate the contents of the FCM in clear natural language sentences with the aid of the RCT. In the following section, the proposed framework is introduced.

\section{Framework}
\label{sec:framework}
If there exists an FCM with nodes, edges and the respective weights, the computable input is available but cannot be interpreted by an unexperienced person. The goal of the framework is to build FCM on raw data in a specific domain and to convert the numerical input of this FCM into natural language with the help of the RCT.\\
First of all, it has to state how the RCT can make use of an FCM. The FCM is usually built based on experiences, whether they are the experts' thoughts and observations or another big amount of data. Therefore, the FCM has to serve as the ED for the RCT, as it provides the data that are used to precisiate variables.\\
In order to build a sentence grammatically, at least a subject and a predicate are required. As a subject, a certain node of the FCM can be utilized, regardless of any context. The predicate has to be defined by the user, depending on the statement that he wants to make. \\

\begin{figure}[ht]
\includegraphics[width=3.5in]{img/framework.png}
\centering
\caption{Framework - Interplay}
\label{fig:framework}
\end{figure}

After all, this is not informative yet; two further elements have to be added to a standard sentence: an object, where another node can be placed, as well as a descriptive word such as an adjective or adverb. It is possible that the edge weight can be used for this purpose, as it provides an idea about the strength of a relationship between the subject and object. Depending on the weight, a certain word is chosen to describe the content.\\
Here, the RCT comes into play. The choice of words is similar to a possibilistic restriction, as outlined in the previous section. Hence, each word has a different membership function according to the connecting passenger frequency. This represents the membership of the word $X$ to the fuzzy connecting frequency set $A$.\\
The totality of word options should cover the whole range of edge weights. The membership degree of each eligible word has to be then evaluated, and the maximum among them is to be chosen. If there are two or more words with the same membership degree, it has to be assumed that these words can be used similarly (i.e., synonyms).\\
A different nature of statements can also be made based on key figures such as the outdegree or the centrality of a node. Additionally, the state of a node at a certain moment during a simulation bears instructive information.\\
The nature of statements strongly depends on the content of the FCM. This context-dependency is what makes it very difficult to define, for example, clear sets of words that can be used for any purpose.\\
In the following section, a use case is presented to prove the validity of the outlined framework.

\section{Use Case for Swiss Airlines: FCM model to describe flight passenger flows}
\label{sec:usecase}
The following use case investigates the passenger flows of an airline. An FCM will be constructed based on connecting passengers (Section \ref{subsec:design}). The result is interpreted by a tool called the interpretation engine or query engine (Section \ref{subsec:queryengine}), which converts the output into natural language statements (Section \ref{subsec:verbalization}). Finally, the generated output is validated by a company's expert (Section \ref{subsec:validation}).\\
\begin{table}[ht]
\begin{tabular*}{0.45\textwidth}{l @{\extracolsep{\fill}} l }
	 \toprule
\emph{Code} & \emph{Airport}  \\\midrule
BCN & Barcelona\\ 
BHX & Birmingham\\ 
BOM & Mumbai\\ 
BSL & Basel\\ 
CDG & Paris Charles de Gaulle\\ 
DAR & Dar es Salaam\\ 
GVA & Geneva\\ 
HAM & Hamburg\\ 
JFK & New York John F. Kennedy\\
LAX & Los Angeles\\ 
LHR & London Heathrow\\ 
LYS & Lyon\\ 
MAN & Manchester\\ 
NRT & Tokyo Narita\\ 
PMI & Palma de Mallorca\\ 
TLV & Tel Aviv\\ 
ZRH & Zurich\\\bottomrule
 \hline
\end{tabular*}
\vspace{0.5cm}
\caption{Airport codes and their respective cities}
\label{tab:airportcodes}
\end{table}

The data for this use case were provided by Swiss (airline code: LX) and contain 1.371.280 passenger records from January 2013. All these passengers travelled on one or more flights operated by LX. Every record contains the following information: origin and destination of the traveler's journey (in separate fields), departure and arrival of the segment, flight number (operator and number in separate fields) and the document number.\\
Table 1 contains the cities and the respective airport codes used in this section.


\subsection{Design of the FCM}
\label{subsec:design}
For designing a new FCM, the specifications of the nodes, the graphs and the properties have to be determined before importing the data. The nodes are depicted as routes. Hence, a connecting passenger from Los Angeles through Zurich to Paris would be depicted as a link from the node LAX-ZRH to the node ZRH-CDG, as shown in Fig. \ref{fig:noderoute}. The intermediate airport is mentioned twice in the node. If an airline has more than only one main hub, it is important to be able to distinguish the routes between the hubs. With the route node specification, this can be assured.

\begin{figure}[ht]
\includegraphics[width=3.5in]{img/route.png}
\centering
\caption{Specification of FCM Nodes as Routes}
\label{fig:noderoute}
\end{figure}

A problem to be solved involves people who connect from another airline to Swiss flights. In any case, it is necessary to identify these passengers as external connectors, as their share on certain routes is substantial for the airline. If they are not identified, they are just counted as regular, non-connection passengers. This is unacceptable; thus, they need to be aggregated in a specific node. Hence, any route that does not belong to the network is eliminated, and the passengers can be identified on the route.\\
Following these arguments, a node will be created for every route in the network of Swiss Airlines. Its name will be composed of the two airports, first the hub and then the destination. If no hub is involved, then the airports will be referred to in alphabetical order. A separate node for all routes outside of the Swiss Airlines network will be built with the name OA, which stands for Other Airline.\\
The specification of the graphs is straightforward: a graph has to be set if a passenger flies on two routes within the same journey. The graph points to the direction of travel.\\
The properties of the FCM elements remain to be specified. Nodes hold the information about their name, that is, the route. A description is added, where the route is mentioned in plain text. Finally, the total number of passengers traveling on this route is provided, including all passengers (connecting or not). Graphs depict the information about the route from where a passenger comes and where a passenger goes. The number of passengers on the graph as well as the ratio between the number of connectors and the total number of passengers on the original route is provided.\\

\subsection{Query Engine}
\label{subsec:queryengine}
In order to interpret the FCM, a query engine has to be defined that generates sentences in natural language. These statements deliver a deeper insight into the FCM and allow a simpler interpretation of the data. For the Swiss Airlines use case, based on the FCM developed using available datasets, two different issues can be considered: 
\begin{itemize}
\item The frequency of connecting passengers between two specific routes in a directed sense, or
\item the frequency of connecting passengers on a specific route to any other.
\end{itemize}
A third issue, the frequency of connecting passengers to a specific route from any other, could not be analyzed with the present data.\\
We mention that the frequency of connecting passengers from one route to another in relation to the total number of passengers on the first route is very low. There are 4.050 different connections between two routes, with a median of 0.2\% and 90\%-percentile of 1.1\%. This means that most of the connections have a very low share, which is why it is important to have an accurate distinction of low shares, whereas the shares above 20\% can be described with very few distinct words.\\
The frequency of connection passengers from one route to any other is, of course, higher than the first one. Here, the median is 25\% and the 90\%-percentile is 43.4\%. Therefore, the range between 20\% and 100\% of the connections can be described with five different adjectives. The remaining range between 0\% and 20\% will be covered with nine different adjectives.\\
All of the adjectives are supposed to express the frequency of a connection and are herewith ordered according to the frequency: \emph{Never, seldom, rarely, occasionally, infrequently, sometimes, frequently, often, regularly, normally, usually, generally, hardly ever,} and \emph{always}. Each of these words has a membership function that defines the word's degree of truth in relation to the share of connecting passengers (\ref{subsubsec:restrictions}). For the adjective \emph{rarely}, this corresponds to:

\begin{equation}
f(rarely)= \begin{cases}
50x + 0.5 & \text{if } 0 \leqslant x < 0.01\\
1 & \text{if } 0.01 \leqslant x < 0.03\\
-50x + 2.5 & \text{if } 0.03 \leqslant x < 0.05
\end{cases}
\end{equation}

When building the sentence, the word with the maximum membership degree is chosen as it best matches the meaning to be given to the sentence. Words with equal degrees can be used synonymously.\\
The frame of the sentences has to be defined for both issues that are considered. For the first issue, passengers who connect between two specific routes in a directed sense, the frame is \newline \emph{"Passengers travelling from (a) to (b) [c] connect to (d)."} \newline In this sentence, \emph{(a)} is the starting point of the journey, which is normally the ending point of the route node, as the hub is always mentioned in first position. Then, \emph{(b)} is the hub, in which the passengers connect to the next flight. \emph{[c]} signifies the adjective, describing the frequency of connections and, finally, \emph{(d)} representing the ending point of the connection.\\
The second issue that is considered in the use case describes the frequency of connecting passengers from a specific route to any other route. Here, the frame is \newline \emph{"Passengers travelling on the route (a) [b] connect to another flight."} \newline For simplicity reasons, only the route \emph{(a)} and the describing word \emph{[b]} are used. Thus, it is a more general statement about connecting passengers; nevertheless, the relevance for analysis is at least as big as that for the first issue.

\subsection{Verbalization of the FCM}
\label{subsec:verbalization}
After having defined and created the FCM, some specific connecting combinations are investigated. The choice of connections should cover the whole variety of long and short haul combinations, as well as journeys through all the three hubs.\\
The first issue is the connection from JFK via ZRH to TLV. The rate of passengers connecting from ZRH-JFK to ZRH-TLV is 0.0298. Based on this value, the truth value is now calculated for every adjective. There are two words with a maximum truth value of 1, that is, rarely and occasionally. These words can be used synonymously, which means that one word can be picked randomly. The sentence to be built would then be \newline \emph{"Passengers travelling from New York to Zurich \textbf{occasionally} connect to Tel Aviv."} \newline The following ten sentences are generated in the same way:
\begin{enumerate}
\item Passengers travelling from Dar es Salaam to Zurich often connect to London Heathrow.
\item Passengers travelling from Palma de Mallorca to Geneva frequently connect to Zurich.
\item Passengers travelling with Other Airlines never connect to the route Zurich-Birmingham.
\item Passengers travelling from Barcelona to Zurich never connect to Tokyo.
\item Passengers travelling from Tokyo to Zurich infrequently connect to Barcelona.
\item Passengers travelling from Barcelona to Basel seldom connect to Hamburg.
\item Passengers travelling from Mumbai to Zurich rarely connect to Manchester.
\item Passengers travelling on the route Zurich-Lyon usually connect to another flight.
\item Passengers travelling on the route Geneva-London Heathrow occasionally connect to another flight.
\item Passengers travelling on the route Zurich-London Heathrow frequently connect to another flight.
\end{enumerate} 
In the next paragraph, these generated sentences are investigated.

\subsection{Validation of the Output}
\label{subsec:validation}
The validation of the sentences mentioned in the previous section is performed by the Head of Channel Management at Swiss International Air Lines Ltd. He is very experienced in the analysis of revenue and passenger data, as he jointly led the development of the passenger revenue information system and later introduced the booking outlook revenue information system. In his present position in Channel Management, he basically deals with prognostics and market behavior, which is however often based on the analysis of the current or a recent situation.\\
In his feedback, he brought up important points regarding the number and choice of adjectives. The seasonal factor is pointed out as an improvement potential; therefore, if the tool is used in a productive environment, it is imperative that it is implemented. The pleasant part of the feedback is that there are no doubts about the correctness of the statements with regards to content. This leads to the conclusion that this use case was successful in proving the feasibility of verbalizing circumstances of a FCM with the aid of RCT.

\section{Lessons learned}
\label{sec:lessons}
The expert's feedback listed three points for improvement:
\begin{itemize}
\item The adjective \emph{never} should be replaced by \emph{do not}
\item The number of adjectives should be reduced
\item The adjective \emph{usually} should be replaced by a different, unspecified word
\end{itemize}
It is true that the word \emph{never} implies a certain finality, saying that passengers do not connect either now or in the future between two routes. Therefore, it makes sense to replace it by a word such as \emph{do not}, which rather makes a statement about the content and does not imply anything for future periods.\\
In order to simplify business decisions, it makes sense to reduce the number of adjectives. Each word would then cause a specific decision to be made. However, the sense of the tool is supposed to depict a big spectrum of adjectives, and hence, the tool can be used for a pre-analysis pointing out fields where a deeper analysis is necessary. The business decisions should then be taken based on these further analyses.\\
In a user's perception, the word \emph{usually} is based on a standard that is either fulfilled or not. Because this standard varies on every route and possibly even in every month, he advises against usage of the adjective. The shares that are used for choosing the word are based on the share of connecting passengers on the original route. Hence, the rating basis varies for every route by definition. However, it is true that the seasonal factor, which is of very high importance in the airline industry, is not considered in the tool. Doing so would imply a heavily higher data load and a time-related dimension, which was not acceptable for the present use case. A workaround would be to create an FCM for every single month and compare the results.

\section{Conclusion and Outlook}
\label{sec:conclusion}
FCMs are applied in many different fields: Medicine \cite{georgopoulos2003}, Ecology \cite{ozesmi2004}, Economy \cite{carvalho2004}, IT Project Management \cite{rodriguez2007} and many more. The interpretation of the underlying FCM is preferentially achieved with If-Then sentences, where a premise leads to a specific result.\\
This kind of interpretation contradicts the concept of fuzziness fundamentally, even though it is possible to formulate the premise as \emph{"If something is true to 0.2, then..."}. However, the basic idea of having concepts and measuring the degree of membership in a fuzzy way is complex, because every fuzzy membership has to be expressed in many different If-Then clauses. With the possibilistic distribution, the RCT provides an instrument that eases the allocation of fuzzy membership degrees to concepts.\\
With the chosen approach of turning FCM output into natural language with the aid of the RCT, the interpretation of the FCM is left to the experts who build the FCM. They lead the definition of the sentence framework, to precisiate the set of words that is used and to specify the associated membership functions. This leads, on the one hand, to the chance that the content is made understandable to a bigger community; on the other hand, there is the risk that some information can be manipulated or withheld, whether by mistake or on purpose.\\
In his paper, Hagiwara \cite{hagiwara1992} points out three major improvement fields for the common FCM: 
\begin{itemize}
\item the proportionality of a relationship between two concepts, 
\item the lack of time delays,
\item the impossibility of representing multiple causality. 
\end{itemize}
The first point is especially interesting when trying to depict customer behavior in relation to the price, which is usually not linear but elastic. It also shows that the full potential of this approach is not yet exploited by far.\\
The introduction of a time dimension on the FCM is to be investigated. This is basically a problem related to the data size, which is intensified further when combined with learning algorithms.\\
On the RCT side, the translation of results from automated pattern recognition into natural language is a field for which many questions are not yet answered, especially in terms of granularity. This would make it possible to evaluate an FCM on an aggregated level and then infer statements about more detailed parts of the FCM.\\
When reading these perspectives, one always has to keep in mind that an FCM (like every model) is an abstraction of the real world. The goal of an abstraction is to create a simplified picture of complex relations. Many of the above-mentioned ideas do not imply any reduction of complexity when modeling the data with current technology. Nevertheless, this should be understood as an inspiration and motivation for future ambitions.

\section*{Acknowledgment}
The authors would like to thank Edi Wolfensberger, who checked the verbalized output of the FCM and gave precious feedback from a user's perspective.

\bibliographystyle{IEEEtran}
\bibliography{fuzzieee2016}

\end{document}


